\documentclass[
    corpo=11.5pt,
    oneside,
    evenboxes,
    tipotesi=triennale,
    stile=classica,
    oldstyle,
    autoretitolo,
    greek,
]{toptesi}
\usepackage[utf8]{inputenc}
\usepackage[T1]{fontenc}
\usepackage{lmodern}
\usepackage{hyperref}
\usepackage{setspace}
\onehalfspacing

\hypersetup{
    pdfpagemode={UseOutlines},
    bookmarksopen,
    pdfstartview={FitH},
    colorlinks,
    linkcolor={blue},
    citecolor={blue},
    urlcolor={blue}
  }
\usepackage{lipsum}

\newtheorem{osservazione}{Osservazione}

\begin{document}\errorcontextlines=9

\begin{ThesisTitlePage}
    \ateneo{Universit\`a degli Studi di Torino}
    \StrutturaDi{Dipartimento di Management}
    \struttura[]{}
    \NomeElaborato{Tesi di laurea triennale}
    \titolo{Lo stato di salute del Calcio}
    \sottotitolo{Fair Play Fiananziario e Superlega}     
    \corsodistudi{Management dell'Informazione e della Comunicazione Aziendale}
    \candidato{Riccardo \textsc{Borgo}}
    \relatore{prof.ssa ~Simona \textsc{Alfiero}}
    \sedutadilaurea{\textsc{Anno~accademico} 2021-2022}
    \logosede{Unito-logo}
\end{ThesisTitlePage}

\tablespagetrue\figurespagetrue
\indici

\mainmatter

%\part{Prima Parte}
\chapter{Introduzione generale}
Questa tesi è stata scritta con lo scopo di analizzare ed evidenziare lo "stato di salute"
dell'ambiente che gravita intorno alle societ\`a di calcio, sopratutto in seguito alla 
pandemia da COVID-19 che ha sensibilmente ridotto i ricavi e le entrate della maggior parte 
delle societ\`a presenti sul mercato, in particolare l'ambiente calcistico è stato tra i 
pi\`u colpiti, vedendosi costretto ad interrompere la propria operativit\`a nei primi mesi del 2020
e, durante i mesi in cui si poteva tornare a giocare, ospitare partite con una capeinza dello stadio ridotta.
Nel susseguirsi dei capitoli si cercher\`a di evidenziare come lo stato di salute dell'ambiente 
calcistico sia peggiorato di anno in anno, con il covid che sicuramente non ha agevolato una possibile 
ripresa, anzi, ha sicueramente portato alla luce una quantità indefinita di problemi finanziari in tutta Europa, nessuna nazione esclusa.

%\blankpagestyle{headings}

%\lipsum[1-2]

\chapter{Il barometro}
\section{Generalit\`a}
\begin{interlinea}{0.87} Il barometro, come dice il nome, serve per
misurare la pesantezza; pi\`u precisamente la pesantezza dell'aria
riferita all'unit\`a di superficie.
\end{interlinea}

\begin{interlinea}{2} Studiando il fenomeno fisico si pu\`o concludere
che in un dato punto grava il peso della colonna d'aria che lo
sovrasta, e che tale colonna \`e tanto pi\`u grave quanto maggiore
\`e la superficie della sua base; il rapporto fra il peso e la base
della colonna si chiama pressione e si misura in once toscane al cubito
quadrato, \cite{tor1}; nel Ducato di Savoia la misura in once al piede
quadrato \`e quasi uguale, perch\'e col\`a usano un piede molto
grande, che \`e simile al nostro cubito.
\end{interlinea}

\subsection{Forma del barometro}
Il barometro consta di un tubo di vetro chiuso ad una estremit\`a e
ripieno di mercurio, capovolto su di un vaso anch'esso ripieno di
mercurio; mediante un'asta graduata si pu\`o misurare la distanza fra
il menisco del mercurio dentro il tubo e la superficie del mercurio
dentro il vaso; tale distanza \`e normalmente di 10 pollici toscani,
\cite{tor1,tor2}, ma la misura pu\`o variare se si usano dei pollici
diversi; \`e noto infatti che gl'huomini sogliono avere mani di
diverse grandezze, talch\'e anche li pollici non sono egualmente
lunghi.
\section{Del mercurio}
Il mercurio \`e un a sostanza che si presenta come un liquido, ma ha il colore
del metallo. Esso \`e pesantissimo, tanto che un bicchiere, che se fosse pieno
d'acqua, sarebbe assai leggiero, quando invece fosse ripieno di mercurio,
sarebbe tanto pesante che con entrambe le mani esso necessiterebbe di essere
levato in suso.

Esso mercurio non trovasi in natura nello stato nel quale \`e d'uopo che sia
per la costruzione dei barometri, almeno non trovasi cos\`i abbondante come
sarebbe necessario.

\setcounter{footnote}{25}

Il Monte Amiata, che \`e locato nel territorio del Ducato%
\footnote{Naturalmente stiamo parlando del Granducato di Toscana.%
\ifclassica\NoteWhiteLine\fi
} del nostro Eccellentissimo et Illustrissimo Signore Granduca di Toscana\footnote{Cosimo IV de' Medici.}, \`e uno dei
luoghi della terra dove pu\`o rinvenirsi in gran copia un sale rosso, che
nomasi \emph{cinabro}, dal quale con artifizi alchemici, si estrae il mercurio
nella forma e nella consistenza che occorre per la costruzione del barometro
terrestre%
\ifclassica
\nota{Nota senza numero\dots

\dots e che va a capo.
}\fi.


La densit\`a del mercurio \`e molto alta e varia con la temperatura come
pu\`o desumersi dalla tabella \ref{t:1}.


Il mercurio gode della sorprendente qualit\`a et propriet\`a, cio\`e che esso
diventa tanto solido da potersene fare una testa di martello et infiggere
chiodi aguzzi nel legname.
\begin{table}[htp]              % crea un floating body col nome Tabella nella
                                % didascalia
\centering                      % comando necessario per centrare la tabella
\begin{tabular}%                % inizio vero e proprio della tabella
{rrrrrr}                        % parametri di incolonnamento
\hline\hline                    % filetti orizzontali sopra la tabella
                                % intestazione della tabella
\multicolumn{3}{c}{\rule{0pt}{2.5ex}Temperatura} % \rule serve per lasciare
& \multicolumn{3}{c}{Densit\`a} \\               % un po' di spazio sopra le parole
    &\unit{\gradi C} & & & $\unit{t/m^3}$ &  \\
\hline%                         % Filetto orizzontale per separare l'intestazione
\hspace*{1.3em}& 0  &  & & 13,8 &  \\   % I numeri sono incolonnati % 
              & 10  &  & & 13,6 &  \\   % a destra; le colonne vuote
              & 50  &  & & 13,5 &  \\   % servono per centrare le colonne
              &100  &  & & 13,3 &  \\   % numeriche sotto le intestazioni
\hline \hline                           % Filetti di fine tabella
\end{tabular}
\caption[Densit\`a del mercurio]{Densit\`a del mercurio. Si pu\`o fare molto meglio usando il pacchetto \textsf{booktabs}.} \label{t:1}  % didascalia con label
\end{table}

%\selectlanguage{italian}

\begin{osservazione}\normalfont
Questa propriet\`a si manifesta quando esso \`e estremamente freddo, come
quando lo si immerge nella salamoia di sale e ghiaccio che usano li maestri
siciliani per confetionare li sorbetti, dei quali sono insuperabili artisti.
\end{osservazione}

Per nostra fortuna, questo grande freddo, che necessita per la confetione de
li sorbetti, molto raramente, se non mai, viene a formarsi nelle terre del
Granduca Eccellentissimo, sicch\'e non vi ha tema che il barometro di mercurio
possa essere ruinato dal grande gelo e non indichi la pressione giusta, come
invece deve sempre fare uno strumento di misura, quale \`e quello che \`e
descritto cost\`i.\cite{duane1964}

\appendix

\begin{thebibliography}{9}
\bibitem{gal} G.~Galilei, {\em Nuovi studii sugli astri medicei}, Manuzio,
        Venetia, 1612.
\bibitem{tor1} E.~Torricelli, in ``La pressione barometrica'', {\em Strumenti
        Moderni}, Il Porcellino, Firenze, 1606.
\bibitem{tor2} E.~Torricelli e A.~Vasari, in ``Delle misure'', {\em Atti Nuovo
        Cimento}, vol.~III, n.~2 (feb. 1607), p.~27--31.
\bibitem{duane1964} Duane J.T., \emph{Learning Curve Approach To Reliability 
		Monitoring}, IEEE Transactions on Aerospace, Vol. 2, pp. 563-566, 1964
\end{thebibliography}




\end{document}

% altri riferimenti da usare come esempi.

\bibitem{chiesa2008} Chiesa S., \emph{Affidabilità, sicurezza e manutenzione 
		nel progetto dei sistemi}, CLUT, gennaio 2008
\bibitem{chiesa2}Chiesa S., Fioriti M., Fusaro R., \emph{On Board System 
		Technological  Level Improvement Effect on UAV MALE}
\bibitem{bigliano2010} Bigliano M., \emph{Sicurezza nell'installazione di un velivolo 
		senza pilota MALE; applicazione di metodologia di Zonal Safety 
		Analysis al velivolo del Progetto SAvE}, Politecnico di Torino, 
		maggio 2010
\bibitem{astrid2012} Chiesa S., Di Meo G.A., Fioriti M., Medici G., Viola N.,
		\emph{ASTRID - Aircraft on board Systems sizing and TRade-off 
		analysis in Initial Design}, Research Bulletin, Warsaw University 
		of Technology, Institute of Aeronautics and Applied Mechanics, 
		p. 1-28, 17-19, ottobre 2012
